% !TEX TS-program = xelatex
% !TEX encoding = UTF-8 Unicode
% !Mode:: "TeX:UTF-8"

\documentclass{resume}
\usepackage{zh_CN-Adobefonts_external} % Simplified Chinese Support using external fonts (./fonts/zh_CN-Adobe/)
% \usepackage{NotoSansSC_external}
% \usepackage{NotoSerifCJKsc_external}
% \usepackage{zh_CN-Adobefonts_internal} % Simplified Chinese Support using system fonts
\usepackage{linespacing_fix} % disable extra space before next section
\usepackage{cite}

\begin{document}
\pagenumbering{gobble} % suppress displaying page number

\name{方缙}

\basicInfo{
  \email{fangjin98@mail.ustc.edu.cn} \textperiodcentered\ 
  \phone{(+86) 181-5566-1676} \textperiodcentered\ 
  \homepage[www.fangjin.site]{www.fangjin.site}
}
 
\section{教育背景}


\datedsubsection{\textbf{中国科学技术大学}}{合肥, 安徽}
\datedsubsection{\textit{硕博连读}\ 计算机科学与技术}{2020.9-至今}
\begin{itemize}
  \item 研究方向为数据中心网络、可编程网络、分布式训练系统、在网计算等
  \item 导师:徐宏力、赵功名
\end{itemize}

\datedsubsection{\textbf{湖南大学}}{长沙, 湖南}
\datedsubsection{\textit{学士}\ 计算机科学与技术}{2016.9-2020.6}
\begin{itemize}
  \item GPA: 3.4 (前15\%)
  \item 保送至中国科学技术大学读研
  \item 获得优秀毕业论文奖
\end{itemize}

\section{论文发表}

\begin{enumerate}
  \item \textbf{J. Fang}, G. Zhao, H. Xu, C. Wu, Z. Yu, \textit{GRID: Gradient Routing with In-network Aggregation for Distributed Training}, IEEE/ACM Transactions on Networking (\textbf{ToN'23}), CCF A
  \item \textbf{J. Fang}, G. Zhao, H. Xu, Z. Yu, B. Shen, X. Li, \textit{GOAT: Gradient Scheduling with Collaborative In-Network Aggregation for Distributed Training}, IEEE/ACM International Symposium on Quality of Service (\textbf{IWQoS'23}), CCF B
  \item \textbf{J. Fang}, G. Zhao, H. Xu, H. Tu, H. Wang, \textit{Reveal: Robustness-Aware VNF Placement and Request Scheduling in Edge Clouds}, Computer Networks (\textbf{ComNet'23}), CCF B
\end{enumerate}

\section{项目经历}

\datedsubsection{\textbf{Simulating network faults with programmable dataplane}}{Suzhou, China}
\datedsubsection{\textit{Main Developer}}{2022.12-2023.9}
\begin{itemize}
  \item Build a user-friendly, multi-backend fault injection system in programmable dataplane
  \item Design a parser generation algorithm to handle flow dependency and load the table entries
  \item Formulate the fault injection point selection problem
  \item Implement several network faults with P4 in TNA and PSA architectures
\end{itemize}

\datedsubsection{\textbf{Accelerating distributed training with programmable switches}}{Zhijiang Lab, Hangzhou, China}
\datedsubsection{\textit{Research Intern}}{2022.6-2022.9}
\begin{itemize}
  \item Improve the in-network aggregation throughput by mitigating the influence of asychronous arrived packets
  \item Design a knapsack-based randomized rounding algorithm for gradient scheduling
  \item Implement a distributed training prototype with Pytorch 
  \item Implement the in-network aggregation logic in Tofino with P4
  \item Reduce the communication overhead of distibuted training tasks by 81.2\%
\end{itemize}

\datedsubsection{\textbf{Developing and testing Alcor, a cloud native SDN platform}}{Futurewei, \textit{Remotely}}
\datedsubsection{\textit{Developer}}{2021.6-2021.9}
\begin{itemize}
  \item Write an automatic building script for large scale deployment with bash
  \item Write an end-to-end test of the virtualization control plane (\href{https://github.com/futurewei-cloud/alcor-control-agent}{ACA}) with C++
  \item Develop grpc thread for pulsar subscribe information (\href{https://github.com/futurewei-cloud/alcor-control-agent/pull/274}{PR \#274}) with C++
\end{itemize}

\datedsubsection{\textbf{Robust-awareness VNF placement in the edge cloud}}{Hefei, China}
\datedsubsection{\textit{Main Developer}}{2021.2-2021.6}
\begin{itemize}
  \item Improve the robustness of edge clouds by limiting the influence of malicious users and failed VNFs
  \item Design a two-phase algorithm to solve the problem of VNF placement and request scheduling
  \item Implement a prototype containing 6 Nvidia Jetson Tx2s and 20 Raspberry Pis with Python
  \item Improve the network throughput by $57\%$ under exisitence the malicious user
\end{itemize}

\datedsubsection{\textbf{Implement a LSTM model based on high-level synthesis}}{Hunan, China}
\datedsubsection{\textit{Main Developer}}{2019.6-2020.1}
\begin{itemize}
  \item Train a LSTM model based on Keras to predict the steam pressure in nuclear power plant reactor
  \item Implement the trained LSTM model with C++ and deploy it into a Pynq-Z2 board
  \item Reduce the inference time by 90x compared with software implementation
  \item \textit{Win the award of Excellent Graduation Thesis of Hunan University}
\end{itemize}

\section{相关专利}

\begin{enumerate}
  \item 赵功名, \textbf{方缙}, 徐宏力, 吴昌博, \textit{PS架构下基于可编程交换机的梯度调度方法和装置}, 已授权
  \item 徐宏力, \textbf{方缙}, 赵功名, 凃化清, 汪海波, \textit{一种边缘云系统中的VNF部署调度方法}, 已公开
\end{enumerate}

\section{获奖情况}

\begin{itemize}[parsep=0.5ex]
  \item \datedline{中国电科十四所国睿奖学金}{2023}
  \item \datedline{Intel P4 中国黑客松优胜奖 (前 25\%)}{2022}
  \item \datedline{一等学业奖学金(博士)}{2022, 2023}
  \item \datedline{一等学业奖学金(硕士)}{2020, 2021}
\end{itemize}
\section{IT 技能}

\begin{itemize}[parsep=0.5ex]
  \item 编程语言: C/C++, Python, P4, C\#, Swift
  \item 开发框架: Pytorch, p4c, eBPF, Mininet
\end{itemize}

\section{其他服务}

\begin{itemize}[parsep=0.5ex]
  \item 审稿: IEEE JSAC, IEEE TNET, COMNET
  \item 助教: COMP6103P Advanced Computer Networking
\end{itemize}


\end{document}
